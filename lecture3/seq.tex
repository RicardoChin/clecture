\section{Program of Sequtial Control}
\begin{frame}<beamer>
    \frametitle{Outline}
    \tableofcontents[currentsection]
\end{frame}

\begin{frame}{Calculate the Area of a circle (1)}
\begin{itemize}
	\item {Available information}
	\begin{itemize}
		\item {radius, $\pi=3.1415$}
	\end{itemize}
	\item {Requirements}
	\begin{itemize}
		\item {Allows user input radius of a circle}
		\item {Calculate its area and print it out}
	\end{itemize}
	\begin{equation}
		a = pi\cdot r^2 \nonumber
	\end{equation}
\end{itemize}
\begin{itemize}
	\item {Let's do it step by step}
\end{itemize}
\end{frame}

\begin{frame}[fragile]{Calculate the Area of a circle (2)}
	\begin{itemize}
		\item {Create a new file named \textbf{main.c}}
		\item {Open it in your editor}
		\item {Fill it as follows:}
	\end{itemize}
\begin{center}
	\begin{lstlisting}[frame=non, xleftmargin=0.2\linewidth]
#include <stdio.h>
int main()
{
    return 0;
}
	\end{lstlisting}
\end{center}
\end{frame}

\begin{frame}[fragile]{Calculate the Area of a circle (3)}
	\begin{itemize}
		\item {Define variables needed}
	\end{itemize}
\begin{center}
	\begin{lstlisting}[frame=non, xleftmargin=0.2\linewidth]
#include <stdio.h>
int main()
{
    float pi = 3.1415;
    float r = 0;
    return 0;
}
	\end{lstlisting}
\end{center}
\end{frame}

\begin{frame}[fragile]{Calculate the Area of a circle (4)}
	\begin{itemize}
		\item {Allows user to input radius,}
	\end{itemize}
\begin{center}
	\begin{lstlisting}[frame=non, xleftmargin=0.2\linewidth]
#include <stdio.h>
int main()
{
    float pi = 3.1415;
    float r = 0;
    scanf("%f", &r);
    return 0;
}
	\end{lstlisting}
\end{center}
\end{frame}

\begin{frame}[fragile]{Calculate the Area of a circle (5)}
	\begin{columns}
		\begin{column}{0.5\linewidth}
	\begin{itemize}
		\item {Allows user to input radius}
	\end{itemize}

	\begin{lstlisting}[frame=non, xleftmargin=0.05\linewidth]
#include <stdio.h>
int main()
{
   float pi = 3.1415;
   float r = 0, area = 0;
   scanf("%f", &r);
   area = r*r*pi;
   return 0;
}
  \end{lstlisting}
	\end{column}
		\begin{column}{0.5\linewidth}
	\begin{itemize}
		\item {The complete program}
	\end{itemize}
	\begin{lstlisting}[frame=non]
#include <stdio.h>
int main()
{
   float pi = 3.1415;
   float r = 0, area = 0;
   scanf("%f", &r);
   area = r*r*pi;
   printf("Area: %f", area);
   return 0;
}
  \end{lstlisting}
	\end{column}
\end{columns}
\end{frame}

\begin{frame}[fragile]{Solve Quadratic Equation (1)}
\begin{itemize}
	\item {Given following equation}
	\item {Allows user input \textit{a}, \textit{b} and \textit{c}}
\end{itemize}
\begin{equation}
	ax^2+bx+c=0 \nonumber
\end{equation}
\begin{itemize}
	\item {Solve \textit{x} out}
\end{itemize}
\end{frame}

\begin{frame}[fragile]{Solve Quadratic Equation (2)}
\begin{itemize}
	\item {The solution for this quadratic equation is well-known}
	\item {Given $b^2-4ac > 0$, we have}
\end{itemize}
\begin{eqnarray}
x_1=\frac{-b+\sqrt{b^2-4ac}}{2a} \nonumber \\
x_2=\frac{-b-\sqrt{b^2-4ac}}{2a} \nonumber 
\end{eqnarray}
\begin{itemize}
	\item {In order to simplify the calculation}
	\item {We have}
\end{itemize}
\begin{eqnarray}
p=\frac{-b}{2a}, & q=\frac{\sqrt{b^2-4ac}}{2a} \nonumber \\
x_1=p+q, & x_2=p-q \nonumber 
\end{eqnarray}

\end{frame}

\begin{frame}[fragile]{Solve Quadratic Equation (3)}
\begin{itemize}
	\item {Let's now think about how to implement it in C}
\end{itemize}
\begin{eqnarray}
p=\frac{-b}{2a}, & q=\frac{\sqrt{b^2-4ac}}{2a} \nonumber \\
x_1=p+q, & x_2=p-q \nonumber
\end{eqnarray}
	\begin{itemize}
		\item {Define variables and user input}
	\end{itemize}
\vspace{-0.25in}
	\begin{columns}
		\begin{column}{0.1\linewidth}
		\end{column}
		\begin{column}{0.9\linewidth}
	\begin{lstlisting}[]
#include <stdio.h>
int main()
{
    float a = 0, b = 0, c = 0, delta = 0;
    float x1 = 0, x2 = 0, p = 0, q = 0;
    printf("Input a, b and c:\n");
    scanf("%f%f%f", &a, &b, &c);
    return 0;
}
	\end{lstlisting}
	\end{column}
	\end{columns}
\end{frame}

\begin{frame}[fragile]{Solve Quadratic Equation (4)}
\vspace{-0.3in}
\begin{eqnarray}
p=\frac{-b}{2a}, & q=\frac{\sqrt{b^2-4ac}}{2a} \nonumber \\
x_1=p+q, & x_2=p-q \nonumber
\end{eqnarray}
\vspace{-0.4in}
	\begin{columns}
		\begin{column}{0.1\linewidth}
		\end{column}
		\begin{column}{0.9\linewidth}
	\begin{lstlisting}[]
#include <stdio.h>
#include <math.h>
int main()
{
    float a = 0, b = 0, c = 0, delta = 0;
    float x1 = 0, x2 = 0, p = 0, q = 0;
    printf("Input a, b and c:\n");
    scanf("%f%f%f", &a, &b, &c);
    delta = b*b - 4*a*c;
    p = -b/(2*a);
    q = sqrt(delta)/(2*a);
    x1 = p + q; x2 = p - q;
    printf("x1=%f, x2=%f\n", x1, x2);
    return 0;
}
	\end{lstlisting}
	\end{column}
\end{columns}
\end{frame}

\begin{frame}[fragile]{Comments}
\vspace{-0.3in}
\begin{eqnarray}
p=\frac{-b}{2a}, & q=\frac{\sqrt{b^2-4ac}}{2a} \nonumber \\
x_1=p+q, & x_2=p-q \nonumber
\end{eqnarray}
\begin{itemize}
	\item {In above example, we did not consider the case}
	\item {${b^2-4ac} < 0$}
	\item {For which, we should output ``no real solution''}
	\item {That means, we should check ${b^2-4ac}$}
	\item {For different case, we give different answer}
	\item {This is where \textcolor{blue}{if}...\textcolor{blue}{else} fits in}
\end{itemize}

\end{frame}

