\section{enum}
\label{sec:enum}
\begin{frame}<beamer>
    \frametitle{Outline}
    \tableofcontents[currentsection]
\end{frame}

\begin{frame}{enum: motivation}
\begin{itemize}
	\item {Sometimes, we feel it is more meaningful }
	\item {with symbols: Janurary, Feburary ,..., December}
	\item {than numbers: 1, 2, ..., 12}
\end{itemize}

\begin{itemize}
	\item {enum allows us to do a kind of correlating}
	\item {Numbers are assigned with readable symbols}
\end{itemize}

\end{frame}

\begin{frame}{enum: definition (1)}
\begin{center}
	\Large{
	   \textcolor{blue}{enum} \textcolor{red}{enumName}\{memb1, memb2, memb3,...\};
	}
\end{center}

\begin{itemize}
	\item {You enumerate all the members' name inside ``\{\}''}
	\item {They are symbols}
	\item {They will be related to integer 0, 1, 2,... automatically}
\end{itemize}

\end{frame}

\begin{frame}[fragile]{enum: definition (2)}
\begin{center}
	\Large{
	   \textcolor{blue}{enum} \textcolor{red}{enumName}\{memb1, memb2, memb3\};
	}
\end{center}
\begin{lstlisting}
enum Month {Jan, Feb, Mar, Apr, May, Jun, Jul, Aug, Sep, Oct, Nov, Dec};
int main()
{
   ...
}
\end{lstlisting}
\begin{itemize}
	\item {You enumerate all the members' name inside ``\{\}''}
	\item {They are symbols}
	\item {They will be related to integer 0, 1, 2,... automatically}
\end{itemize}
\end{frame}


\begin{frame}[fragile]{enum: how to use it}
\begin{lstlisting}
#include <stdio.h>
enum Month {Jan, Feb, Mar, Apr, May, Jun, Jul, Aug, Sep, Oct, Nov, Dec};
int main()
{
   enum Month m;
   m = Feb;
   printf("Month is: %d\n", m);
   return 0;
}
\end{lstlisting}
[Output]
\begin{lstlisting}
Month is: 1
\end{lstlisting}
\begin{itemize}
	\item {\textbf{Feb} is a symbol instead of a string}
	\item {They will be related to integer 0, 1, 2,... automatically}
\end{itemize}

\end{frame}

\begin{frame}[fragile]{enum: learn by example (1)}
\begin{columns}
\begin{column}{0.52\linewidth}
\begin{lstlisting}
#include <stdio.h>
enum Week {Mon=1, Tue=1, Wed=3,
 Thu=5, Fri, Sat=4, Sun};
int main()
{
   enum Week wk;
   wk=Wed;
   printf("Wed: %d\n", wk);
   wk=Fri;
   printf("Fri: %d\n", wk);
   wk=Sun;
   printf("Sun: %d\n", wk);
   return 0;
}
\end{lstlisting}
\end{column}
\begin{column}{0.43\linewidth}
[Output]
\begin{lstlisting}
Wed: ?
Fri: ?
Sun: ?
\end{lstlisting}
\end{column}
\end{columns}
\end{frame}

\begin{frame}[fragile]{enum: learn by example (2)}
\begin{columns}
\begin{column}{0.52\linewidth}
\begin{lstlisting}
#include <stdio.h>
enum Week {Mon=1, Tue=1, Wed=3,
 Thu=5, Fri, Sat=4, Sun};
int main()
{
   enum Week wk;
   wk=Wed;
   printf("Wed: %d\n", wk);
   wk=Fri;
   printf("Fri: %d\n", wk);
   wk=Sun;
   printf("Sun: %d\n", wk);
   return 0;
}
\end{lstlisting}
\end{column}
\begin{column}{0.43\linewidth}
[Output]
\begin{lstlisting}
Wed: 3
Fri: 6
Sun: 5
\end{lstlisting}
\end{column}
\end{columns}
\begin{itemize}
	\item {Can you figure out why??}
	\item {This way is valid, but NOT \textcolor{red}{suggested}}
\end{itemize}
\end{frame}

\begin{frame}[fragile]{enum: learn by example (2)}
\begin{columns}
\begin{column}{0.52\linewidth}
\begin{lstlisting}
#include <stdio.h>
enum Week {Mon=1, Tue, Wed,
 Thu, Fri, Sat, Sun};
int main()
{
   enum Week wk;
   wk=Wed;
   printf("Wed: %d\n", wk);
   wk=Fri;
   printf("Fri: %d\n", wk);
   wk=Sun;
   printf("Sun: %d\n", wk);
   return 0;
}
\end{lstlisting}
\end{column}
\begin{column}{0.43\linewidth}
[Output]
\begin{lstlisting}
Wed: 3
Fri: 5
Sun: 7
\end{lstlisting}
\end{column}
\end{columns}
\begin{itemize}
	\item {This is the right way}
\end{itemize}
\end{frame}

\begin{frame}[fragile]{enum: learn by example (3)}
\begin{columns}
\begin{column}{0.52\linewidth}
\begin{lstlisting}
#include <stdio.h>
enum Week {Mon=1, Tue, Wed,
 Thu, Fri, Sat, Sun};
typedef enum Week WkType;
int main()
{
   WkType wk;
   wk=Wed;
   printf("Wed: %d\n", wk);
   wk=Fri;
   printf("Fri: %d\n", wk);
   wk=Sun;
   printf("Sun: %d\n", wk);
   return 0;
}
\end{lstlisting}
\end{column}
\begin{column}{0.43\linewidth}
[Output]
\begin{lstlisting}
Wed: 3
Fri: 5
Sun: 7
\end{lstlisting}
\end{column}
\end{columns}
\begin{itemize}
	\item {You can use ``\textcolor{blue}{typedef}'' to save up your coding efforts}
\end{itemize}
\end{frame}
