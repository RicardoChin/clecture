\begin{frame}[fragile]{Opening Discussion}
\begin{itemize}
	\item {Given we have following informatio for \textit{40} students}
	\begin{enumerate}
		\item {student number}
		\item {name}
		\item {age}
		\item {gender}
		\item {height}
		\item {GPA}
	\end{enumerate}
	\item {We now want to build records for all the students}
\end{itemize}
\begin{lstlisting}[linewidth=0.7\linewidth, xleftmargin=0.05\linewidth]
int main()
{
    char std1nm[64], std2nm[64], ...;
    char std1nb[11], std2nb[11], ...;
    int  std1ag, std2ag, ...;
    char std1gd[5], std1gd[5], ...;
    ...
}
\end{lstlisting}
\end{frame}

\section{struct}
\label{sec:structs}
\begin{frame}<beamer>
    \frametitle{Outline}
    \tableofcontents[currentsection]
\end{frame}

\begin{frame}[fragile]{Composite Data Types}
\begin{itemize}
	\item {It is valid/OK to do it in the way we learned}
	\item {However, it is not convenient}
	\item {C provides us the way to extend current data types}
	\item {We can combine primitive types into one type}
\end{itemize}
\begin{columns}
\begin{column}{0.2\linewidth}
\end{column}
\begin{column}{0.4\linewidth}
\begin{lstlisting}
struct STD {
   char stdNm[64];
   char stdNb[11];
   int  age;
   char gender[5];
};
\end{lstlisting}
\end{column}
\begin{column}{0.2\linewidth}
\end{column}
\end{columns}
\begin{itemize}
	\item {``\textcolor{blue}{struct} STD'' is a new data type}
	\item {Its role is similar as \textcolor{blue}{int}, or \textcolor{blue}{float},...}
\end{itemize}
\end{frame}

\begin{frame}[fragile]{struct: grammar of definition}
\begin{center}
	\Large{
	\textcolor{blue}{struct} \textcolor{red}{structTag} \{ \\
	  \textcolor{blue}{type1} member1; \\
	  \textcolor{blue}{type2} member2; \\
	   ... \\
	  \textcolor{blue}{typeN} memberN; \};
	}
\end{center}
\begin{itemize}
	\item {Keyword ``\textcolor{blue}{struct}'' is required, it tells C you are going to define a composite type}
	\item {\textcolor{red}{structTag} gives a \textbf{unique} tag for this new type}
	\item {You list all the memebers and their corresponding types}
	\item {``\textcolor{red}{;}'' is required at the end}
	\item {Keep in your mind, you define a \textbf{type} instead of a variable/constant}
\end{itemize}
\end{frame}

\begin{frame}[fragile]{struct: define variable of composite type (1)}
\begin{center}
	\Large{
	\textcolor{blue}{struct} \textcolor{red}{structTag} \textbf{record};
	}
\end{center}
\begin{itemize}
	\item {Keyword ``\textcolor{blue}{struct}''  and \textcolor{red}{structTag} are required}
	\item {``record'' is the variable name of \textcolor{red}{structTag} type}
\end{itemize}
\end{frame}

\begin{frame}[fragile]{struct: define variable of composite type (2)}
\begin{center}
	\Large{
	\textcolor{blue}{struct} \textcolor{red}{structTag} \textbf{record};
	}
\end{center}
\begin{itemize}
	\item {Keyword ``\textcolor{blue}{struct}''  and \textcolor{red}{structTag} are required}
	\item {``record'' is the variable name of \textcolor{red}{structTag} type}
\end{itemize}
\begin{lstlisting}
struct STD {
   char stdNm[64];
   char stdNb[11];
   int  age;
   char gender[5];
};
int main()
{
   struct STD record;
   struct STD stds[40];
}
\end{lstlisting}
\end{frame}

\begin{frame}[fragile]{struct: initialize variable of composite type (1)}
\begin{itemize}
	\item {Each memember in the composite type variable is treated as a variable}
	\item {They are visited via ``var.member1''}
\end{itemize}
\begin{lstlisting}[linewidth=0.7\linewidth, xleftmargin=0.05\linewidth]
struct STD {
   char stdNm[64];
   char stdNb[11];
   int  age;
   char gender[5];
};
int main()
{
   struct STD record;
   strcpy(record.stdNm, "Min Li");
   strcpy(record.stdNb, "11201522031");
   record.age = 20;
   strcpy(record.gender, "male");
}
\end{lstlisting}
\end{frame}

\begin{frame}[fragile]{struct: initialize variable of composite type (2)}
\vspace{-0.15in}
\begin{lstlisting}
#include <stdio.h>
#include <string.h>
struct STD {
   char stdNm[64];
   char stdNb[11];
   int  age;
   char gender[5];};
int main()
{
   struct STD std;
   strcpy(std.stdNm, "Min Li");
   strcpy(std.stdNb, "22031");
   std.age = 20;
   strcpy(std.gender, "male");
   printf("Name: %s\n", std.stdNm);
   printf("Numb: %s\n", std.stdNb);
   printf("Age: %d\n", std.age);
   printf("Gender: %s\n", std.gender);
}
\end{lstlisting}
\end{frame}

\begin{frame}[fragile]{struct: exmaple (1)}
\begin{itemize}
	\item {Please build a struct type for date (Year, month and day)}
	\item {Work out which day it is of the year}
	\begin{enumerate}
		\item {We need \textcolor{blue}{struct} type to keep date inform}
		\item {We need to calculate which day of the year is}
		\item {It depends on year (whether it is a leap year)}
		\item {Depends on the month}
		\item {Depends on the date}
	\end{enumerate}
\end{itemize}
\begin{center}
	\Large{
	   5 minutes to think about it...
	}
\end{center}
\end{frame}

\begin{frame}[fragile]{struct: exmaple (2)}

[General procedure]
\begin{enumerate}
	\item {Accept input, save the information to a date structure}
	\item {Check whether the year is leap year or not}
	\item {Check which month it is}
	\item {We need an array to keep the days of months}
\end{enumerate}

\end{frame}

\begin{frame}[fragile]{struct: exmaple (3)}

[General procedure in more detail]
\begin{enumerate}
	\item {Define a date struct}
	\item {Accept input, save the information to a date structure}
	\item {Initialize of days of months (12 months)}
	\item {If it is leap year and date.month $>=$ 3}
	\item {~~Plus 1 day to the total}
	\item {End-If}
	\item {For i from 1 to (date.month-1)}
	\item {~~sum up days of months before current month}
	\item {End-for}
\end{enumerate}

\end{frame}

\begin{frame}[fragile]{struct: exmaple (4)}
\vspace{-0.22in}
\begin{columns}
\begin{column}{0.48\linewidth}
\begin{lstlisting}
struct DATE {
   int day, month, year;
};

int main()
{
   struct DATE date;
   int dyMonth[]={31,28,31,
   30,31,30,31,31,
   30,31,30,31};
   int i = 1, dayth = 0;
   printf("Year:");
   scanf("%d", &date.year);
   printf("Month:");
   scanf("%d", &date.month);
   printf("Day:");
   scanf("%d", &date.day);
\end{lstlisting}
\end{column}
\begin{column}{0.48\linewidth}
\begin{lstlisting}[firstnumber=17]
   if(isLeap(date.year))
   {
      dayth += 1;
   }
   for(;i<date.month;i++)
   {
      dayth+= dyMonth[i-1];
   }
   dayth += date.day;
   return 0;
}
\end{lstlisting}
\end{column}
\end{columns}
\vspace{-0.15in}
\textcolor{red}{Is there anything wrong?? Two mistakes!!}
\end{frame}

\begin{frame}[fragile]{struct: exmaple (5)}
\vspace{-0.15in}
\begin{columns}
\begin{column}{0.48\linewidth}
\begin{lstlisting}
struct DATE {
   int day, month, year;
};
int isLeap(int year)
{  if(year%4==0) {
      if(year%400==0){
         return 1;
      }else if(year%100=0){
         return 0;
      }
      return 1;
   }else{
      return 0;
   }//end-if-else
}//end-isLeap
int main()
{  struct DATE date;
   int dyMonth[]={31,28,31,
   30,31,30,31,31,
   30,31,30,31};
\end{lstlisting}
\end{column}
\begin{column}{0.48\linewidth}
\begin{lstlisting}[firstnumber=21]
   int i = 1, dayth = 0;
   printf("Year:");
   scanf("%d", &date.year);
   printf("Month:");
   scanf("%d", &date.month);
   printf("Day:");
   scanf("%d", &date.day);
   if(isLeap(date.year)&&date.month>2)
   {
      dayth += 1;
   }
   for(;i<date.month;i++)
   {
      dayth+= dyMonth[i-1];
   }
   dayth += date.day;
   return 0;
}
\end{lstlisting}
\end{column}
\end{columns}
\end{frame}


\begin{frame}[fragile]{struct: size of the struct type (1)}
\begin{itemize}
	\item {Now let's consider another problem}
	\item {What is the size (bytes occupied) of struct type variable}
\end{itemize}
\begin{columns}
\begin{column}{0.48\linewidth}
\begin{lstlisting}
struct DATE {
   int day, month, year;
};
struct STD{
   char Name[10];
   int age;
   char gender[6];
};
int main()
{  
   printf("%d\n", sizeof(struct DATE));
   printf("%d\n", sizeof(struct STD));
   return 0;
}
\end{lstlisting}
\end{column}
\begin{column}{0.48\linewidth}
[Output]
\begin{lstlisting}
12
24
\end{lstlisting}
\begin{itemize}
	\item {Can you figure out why?}
\end{itemize}
\end{column}
\end{columns}
\end{frame}

\begin{frame}[fragile]{struct: size of the struct type (2)}
\begin{itemize}
	\item {Now let's consider another problem}
	\item {What is the size (bytes occupied) of struct type variable}
\end{itemize}
\begin{columns}
\begin{column}{0.48\linewidth}
\begin{lstlisting}
struct DATE {
   int day, month, year;
};
struct STD{
   char Name[10];
   int age;
   char gender[6];
};
int main()
{  
   printf("%d\n", sizeof(struct DATE));
   printf("%d\n", sizeof(struct STD));
   return 0;
}
\end{lstlisting}
\end{column}
\begin{column}{0.48\linewidth}
[Output]
\begin{lstlisting}
12
24
\end{lstlisting}
\begin{itemize}
	\item {\textcolor{green}{Name} will be given with \textcolor{red}{12} bytes instead of 10}
	\item {\textcolor{green}{gender} will be given with \textcolor{red}{8} bytes instead of 6}
	\item {For the convevience of memory allocation}
	\item {This could be different from one compiler to another}
\end{itemize}
\end{column}
\end{columns}
\end{frame}


\begin{frame}[fragile]{struct: \textcolor{blue}{typedef} to save code (1)}
\begin{itemize}
	\item {In ``struct STD'',  ``\textcolor{blue}{struct}'' has been repeated everywhere}
	\item {We can use ``\textcolor{blue}{typedef}'' to save up our typing}
\end{itemize}
\vspace{-0.15in}
\begin{columns}
\begin{column}{0.48\linewidth}
\begin{lstlisting}
struct DATE {
   int day, month, year;
};
struct STD{
   char Name[10];
   int age;
   char gender[6];
};
typedef struct STD StdType;
typedef struct DATE DatType;
int main()
{  
   DatType date;
   StdType std;
   printf("%d\n", sizeof(DatType));
   ...
}
\end{lstlisting}
\end{column}
\begin{column}{0.48\linewidth}
\begin{itemize}
	\item {During compiling stage}
	\item {``StdType'' is replaced by ``\textcolor{blue}{struct} STD''}
\end{itemize}
\end{column}
\end{columns}
\end{frame}

\begin{frame}[fragile]{struct: \textcolor{blue}{typedef} to save code (2)}
\begin{itemize}
	\item {You can apply \textcolor{blue}{typedef} to any type}
\end{itemize}
\vspace{-0.05in}
\begin{lstlisting}
#include <stdio.h>
typedef unsigned int uint;

int main()
{
   uint a = 32768;
   printf("%d\n", a);
   printf("%d\n", sizeof(uint));
   return 0;
}
\end{lstlisting}
\begin{itemize}
	\item {During compiling stage}
	\item {``uint'' is replaced by ``\textcolor{blue}{unsigned int}''}
	\item {You actually give a \textbf{nickname} to the type by \textcolor{blue}{typedef}}
\end{itemize}
\end{frame}


