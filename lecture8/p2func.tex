\section{Pointer to Function}
\label{sec:p2func}
\begin{frame}<beamer>
    \frametitle{Outline}
    \tableofcontents[currentsection]
\end{frame}

\begin{frame}{An Overview: Motivation (1)}
\begin{figure}
	\includegraphics[width=0.40\linewidth]{pics/sqrt_integral.png}
	\caption{Numerical integral of $\sqrt{x}$}
\end{figure}

\begin{itemize}
	\item {Given two functions to perform the numerical integral}
	\item {$f(x)=\sqrt{x}$, $g(x)=cos(x)$}
	\item {$\int_a^bf(x)dx = ?$, $\int_a^bg(x)dx = ?$}
\end{itemize}

\end{frame}

\begin{frame}[fragile]{An Overview: Motivation (2)}
\begin{itemize}
	\item {Define dx=0.05, given \textcolor{blue}{a} and \textcolor{blue}{b}}
	\item {We can calculate integral of $\sqrt{x}$ when  $x \in [a, b]$}
\end{itemize}
\begin{lstlisting}[xleftmargin=0.08\linewidth, linewidth=0.9\linewidth]
#include <math.h>
#include <stdio.h>
float intSqrt(float dx, float a, float b){
  float s = 0, x = a;
  while(x < b){
    s += sqrt(x)*dx;
    x += dx;
  }
  return s;
}
float intCos(float dx, float a, float b){
  float s = 0, x = a;
  while(x < b){
    s += cos(x)*dx;
    x += dx;
  }
  return s;
}
\end{lstlisting}
\end{frame}

\begin{frame}[fragile]{An Overview: Motivation (3)}
\begin{itemize}
	\item {Define dx=0.05, given \textcolor{blue}{a} and \textcolor{blue}{b}}
	\item {We can calculate integral of $\sqrt{x}$ when $x \in [a, b]$}
\end{itemize}
\begin{lstlisting}[xleftmargin=0.08\linewidth, linewidth=0.9\linewidth,firstnumber=19]
void main()
{
  float a = 1.0, b = 5.0, dx = 0.05, s = 0;
  char funcName[8] = "";
  scanf("%s", &funcName);
  if(strcmp(funcName, "sqrt") == 0){
     s = intSqrt(dx, a, b);
  }else if(strcmp(funcName, "sin") == 0){
     s = intSin(dx, a, b);  
  }else if(strcmp(funcName, "cos") == 0){
     s = intCos(dx, a, b);    
  }
  printf("Integral is: %f\n", s);
}
\end{lstlisting}
\end{frame}

\begin{frame}[fragile]{An Overview: Motivation (4)}
\begin{itemize}
	\item {Define dx=0.05, given \textcolor{blue}{a} and \textcolor{blue}{b}}
	\item {We can calculate integral of $\sqrt{x}$ $x \in [a, b]$}
\end{itemize}
\begin{lstlisting}[xleftmargin=0.08\linewidth, linewidth=0.9\linewidth,firstnumber=19]
void main()
{
  float (*fun_ptr)(float dx, float a, float b);
  float a = 1.0, b = 5.0, dx = 0.05, s = 0;
  char funcName[8] = "";
  scanf("%s", &funcName);
  if(strcmp(funcName, "sqrt") == 0){
     func_ptr = &intSqrt;
  }else if(strcmp(funcName, "sin") == 0){
	 func_ptr = &intSin;  
  }else if(strcmp(funcName, "cos") == 0){
	 func_ptr = &intCos;
  }
  s = (*func_ptr)(dx, a, b);
  printf("Integral is: %f\n", s);
}
\end{lstlisting}
\end{frame}

\begin{frame}[fragile]{Function Pointer: the declaration (1)}
\begin{center}
\Large{
   \textcolor{blue}{type0} (\textcolor{red}{*function\_pointer\_name})(\textcolor{blue}{type1} p1, \textcolor{blue}{type2} p2);
}
\end{center}
\begin{itemize}
	\item {Given a function in the same form }
\end{itemize}
\begin{center}
\Large{
\textcolor{blue}{type0} \textcolor{red}{fun1}(\textcolor{blue}{type1} p1, \textcolor{blue}{type2} p2);
}

\textcolor{red}{*function\_pointer\_name} = \&\textcolor{red}{fun1};
\end{center}

\begin{lstlisting}[xleftmargin =0.05\linewidth, linewidth=0.7\linewidth]
#include <stdio.h>
int add(int a, int b){
   return a+b;
}
int main(){
   int (*pfun)(int a, int b) = NULL;
   int a = 5, b = 8, r = 0;
   pfun = &add;
   r = pfun(a, b);
   printf("r = %d\n", r);
   return 0;
 }
\end{lstlisting}
\end{frame}

\begin{frame}[fragile]{Function Pointer: the declaration (2)}
\begin{figure}
\begin{center}
	\includegraphics[width=0.48\linewidth]{figs/pfunc.pdf}
\end{center}
\end{figure}
\begin{lstlisting}[xleftmargin =0.05\linewidth, linewidth=0.7\linewidth]
#include <stdio.h>

int add(int a, int b){
   return a+b;
}
int main(){
   int (*pfun)(int a, int b) = NULL;
   int a = 5, b = 8, r = 0;
   pfun = &add;
   r = pfun(a, b);
   printf("r = %d\n", r);
   return 0;
 }
\end{lstlisting}
\end{frame}

\begin{frame}[fragile]{Function Pointer: the declaration (3)}

\begin{lstlisting}[xleftmargin =0.05\linewidth, linewidth=0.86\linewidth]
#include <stdio.h>

int add(int a, int b){
   return a+b;
}
int main()
{
   int (*pfun)(int a, int b) = NULL;
   int a = 5, b = 8, r = 0;
   pfun = &add;
   r = pfun(a, b);
   printf("size of pointer: %d\n", sizeof(pfun));
   printf("r = %d\n", r);
   return 0;
 }
\end{lstlisting}
\end{frame}
